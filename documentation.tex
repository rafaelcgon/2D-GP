\documentclass[12pt,a4paper]{article}%[a4paper,portuguese,12pt,pdftex]{article} %article
\usepackage{ucs}
\usepackage{amsfonts}		 		% usar fontes p/ REAL,IMAGINARIO,etc
\usepackage{indentfirst}			% espacamento no 1o. paragrafo
\usepackage[utf8x]{inputenc} 
\usepackage{amssymb,amsmath}			% pacote simbolos e matematico AMS
\usepackage{textcomp}
\usepackage{natbib}				% estilo de citacao no texto
\usepackage{wasysym}
\usepackage{fancyhdr}
\renewcommand{\baselinestretch}{1.5}		% espaco entre linhas
\topmargin -1cm					% margem superior
\oddsidemargin 0.5cm				% margem esquerda
\textwidth 15cm					% largura da pagina
\textheight 23.7cm				% altura da pagina
\usepackage{color}
\usepackage{lscape}
\usepackage{graphicx}
\usepackage{epsfig}

\title{Learning velocity fields from sparse data using Gaussian Process}
\author{Rafael Gon\c{c}alves}

\begin{document}

\section{Interpolation of drifters' tracks and estimative of velocity}

The drifters from the LASER experiment provided positions at intervals of approximatly 5 minutes. 
These intervals may vary largelly due to the state of the GPS sensors (e.g. low battery) or due to 
the sea state, as waves might flip the undrogued drifters, interrupting the GPS transmission.
The interpolation of the drifters' positions is carried out in order to have all drifters' tracks in 
the same regular time frame. Here, two different approaches were considered for the interpolation: the 
B-spline function and Gaussian Process regression. In both cases, the time series of Longitude and Latitude 
recorded by each drifter is interpolated in a time frame with 10 minutes resolution.

In our first approach, we used cubic B-splines. The formalism of the algorithm applied for the B-spline 
interpolation is described in \cite{Dierckx1975,Dierckx1981,Dierckx1982,Dierckx1993}. The algorithm uses a 
smoothing condition $S$, for which the interpolated curve must satisfy:

\begin{equation}
 \sum_{i=0}^N((y_i - g_i))^2 \le S
\end{equation}

where $g$ is the smoothed interpolation of $y$, and $N$ is the number of data points used. 
The amount of smoothing alowed by $S$ should be enough to smooth out spikes that might occur 
due to the GPS positioning uncertainty \citep{Haza20?}, and also to avoid overfitting and ill-posed 
problems \citep{Dierckx1981}. Here, the time series of Latitude and Longitude were 
interpolated with $S=0.001$. One example of an interpolated drifter track is shown on Figure \ref{fig-track}.
The B-spline algorithm enables the computation of its derivatives (see \cite{Dierckx1975}). 
The first derivative of the Longitude and Latitude B-splines were used to compute the zonal and meridional 
components of the Lagrangian velocities.

In the Gaussian process regression approach, the time series of Longitude and Latitude of each drifter are 
treated as separate Gaussian processes, so that the hyperparameters of their covariance functions can 
be set with different values. Here, we apply the squared exponential covariance function:

\begin{equation}
 k(t_i,t_j) = \sigma^2 exp(-\frac{(t_i-t_j)^2}{2\tau^2}) % + \sigma_N\delta_{ij}
\end{equation}

where t is time, $\sigma$ is the variance and $\tau$ is a decorrelation time scale. %, and $\sigma_N\delta_{ij}$ 
 

\begin{figure}
\noindent\includegraphics[width=36pc]{drifter_tracks.png}
\caption{The positions recorded by one drifter over 2 days (black dots) and interpolated 
positions using the cubic B-spline (blue dots) and Gaussian Process (red dots).}
\label{fig-track}
\end{figure}

\begin{figure}
\noindent\includegraphics[width=36pc]{velocity_components.png}
\caption{Lagrangian velocity components estimated from the cubic B-spline interpolation (blue) 
and from the Gaussian Process (red).}
\label{fig-track}
\end{figure}


\end{document}


%1] P. Dierckx, "An algorithm for smoothing, differentiation and
%   integration of experimental data using spline functions",
%   J.Comp.Appl.Maths 1 (1975) 165-184.
%.. [2] P. Dierckx, "A fast algorithm for smoothing data on a rectangular
%   grid while using spline functions", SIAM J.Numer.Anal. 19 (1982)
%   1286-1304.
%.. [3] P. Dierckx, "An improved algorithm for curve fitting with spline
%   functions", report tw54, Dept. Computer Science,K.U. Leuven, 1981.
%.. [4] P. Dierckx, "Curve and surface fitting with splines", Monographs on
%   Numerical Analysis, Oxford University Press, 1993.
